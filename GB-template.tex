%!TEX program = xelatex
\documentclass{GB-template}
\usepackage{textcomp}
\usepackage{pifont}
\usepackage{amsmath,amssymb,amsfonts}
\usepackage{enumitem}
\usepackage{tabularx}
\usepackage{multirow}
\setlist[enumerate]{nosep, label={\alph*)},left=2\ccwd}
% use parameters
\ICS{ICS 01.140.20}
\fenlei{CCS A 14}
\biaozhunhao{GB/T 7713.2~\rule[.3\ccwd]{.8em}{.5pt}~2022}
\daitibiaozhunhao{部分代替:GB/T 7713~\rule[.3\ccwd]{.8em}{.5pt}~1987}
\name{学术论文编写规则}
\enname{Presentation of academic papers}
\conformity{}
\department{\parbox{11\ccwd}{国家市场监督管理总局\\
国家标准化管理委员会}}
\fabudate{2022-12-30}
\shishidate{2023-07-01}
\setmainfont{texgyretermes}[
  UprightFont = *-regular ,
  BoldFont = *-bold ,
  ItalicFont = *-italic ,
  BoldItalicFont = *-bolditalic ,
  Extension = .otf ,
  Scale = 1.0 ]
 \setCJKfamilyfont {zhei}{FandolHei-Regular.otf} 
 \newcommand{\hei}{\CJKfamily{zhei}}
\newcommand{\SubSectionStart}{\begin{enumerate}[label=\thesection.\arabic*,wide, labelwidth=!, labelindent=0pt]}
\newcommand{\SubSection}[1]{\item #1}
\newcommand{\SubSectionEnd}{\end{enumerate}}

   \setcounter{tocdepth}{1}
   
   \usepackage{pifont}
\usepackage{varwidth}
\usepackage{tcolorbox}
\tcbuselibrary{skins,xparse,breakable,xparse} 

\newtcolorbox{mybox}[2]{enhanced,breakable,
before skip=2mm,after skip=2mm,
colback=white!40,colframe=#2!30!black,boxrule=0.3mm,rightrule=0.3mm,
attach boxed title to top center={xshift=0cm,yshift*=1mm-\tcboxedtitleheight},
varwidth boxed title*=-3cm,
boxed title style={frame code={
\path[fill=#2!30!black]
([yshift=-1mm,xshift=-1mm]frame.north west)
arc[start angle=0,end angle=180,radius=1mm]
([yshift=-1mm,xshift=1mm]frame.north east)
arc[start angle=180,end angle=0,radius=1mm];
\path[draw=black,line width=1pt,left color=#2!1!white,right color=#2!1!white,
middle color=#2!1!white]
([xshift=-2mm]frame.north west) -- ([xshift=2mm]frame.north east)
[rounded corners=1mm]-- ([xshift=1mm,yshift=-1mm]frame.north east)
-- (frame.south east) -- (frame.south west)
-- ([xshift=-1mm,yshift=-1mm]frame.north west)
[sharp corners]-- cycle;
},interior engine=empty,
},
title=#1,coltitle=black,fonttitle=\large\sffamily}
   
\begin{document}
% preamble.tex 里面是前言部分。
\begin{preamble}{重排自序}
\begin{mybox}{重排说明 }{gray}

最近不少用户找我要这个标准《学术论文编写规则》的可编辑版本,在网上找到的都是图片,我就特此制作了这个版本。内容校对了几次,可能还有疏漏。仅供参考。

如果有问题,可以加我 微信:t314159265 , 感谢各位!

大家还是查看官网的版本,地址:\url{https://std.samr.gov.cn/gb/search/gbDetailed?id=F159DFC2A91247EFE05397BE0A0AF334}

\end{mybox}
\end{preamble}

\begin{preamble}{前言}
  本文件按照GB/T1.1-2020《标准化工作导则 第1部分:标准化文件的结构和起草规则》的规定起草。

本文件是GB/T7713的第2部分。GB/T7713已经发布了以下部分:

——第1部分:学位论文编写规则:

——第2部分:学术论文编写规则:

——第3部分:科技报告编写规则。

本文件部分代替GB/T7713-1987《科学技术报告、学位论文和学术论文的编写格式》,与GB/T7713一1987相比,除结构调整和编辑性改动外,主要技术变化如下:

\begin{enumerate}[label={\alph*)}]
  \item
 将其适用范围扩展至印刷版、缩微版、电子版等所有传播形式的学术论文(见第 1章);

  \item   将“引言”更改为“范围”,更改了相关表述(见第1章,1987 年版的第1章);

  \item   将“定义”更改为“术语和定义”,删除了与学术论文编写规则无关的术语和定义,更改了“学术论文”的定义,增加了“正文部分”“参考文献”的定义(见第3章,1987 年版的第2章);

  \item  将“编写要求”“编写格式”更改为“编排格式”(见第5章,1987 年版的第3章第4章);

  \item  将“前置部分”“主体部分”“附录”“结尾部分”更改为“组成部分”“编排格式”(见第 4 章第5章,1987年版的第5章、第6章、第7章、第8章);

  \item  “组成部分”的更改:关于题名字数,将“题名一般不宜超过 20 字”更改为“为便于交流和利用题名应简明,一般不宜超过 25字”;关于摘要,将“中文摘要一般不宜超过 200$\sim$300字;外文摘要不宜超过 250 个实词”更改为“中文摘要的字数,原则上应与论文中的成果多少相适应,在般情况下,报道性摘要以400字左右报道/指示性摘要以300 字左右指示性摘要以150字左右为宜。中文摘要、外文摘要内容宜对应,为利于国际交流,外文摘要可以比中文摘要包含更多信息”,删除了“除了实在无变通办法可用以外,摘要中不用图、表、化学结构式”;在“其他项目”中,增加了学术论文前置部分要求、建议或允许标注的项目,如基金名称及项目编号、收稿日期、引用本论文的参考文献格式,论文增强出版的元素以及相关声明等(见4.2.1、4.2.3、4.2.5,1987年版的5.5.1、5.7.4、5.7.5);

  \item  在“编排格式”中,选列了学术论文的编号、量和单位、插图、表格数字、数学式、注释、科学技术名词的规范化要点及示例(见5.2、5.3、5.4、5.5、5.6、5.7、5.8、5.9);

  \item   删除了附录 A“封面示例”和附录 B“相关标准”,增加了规范性附录 A“学术论文的构成要素”(见附录A.1987年版的附录A附录B)。

\end{enumerate}

请注意本文件的某些内容可能涉及专利。本文件的发布机构不承担识别专利的责任。

本文件由全国信息与文献标准化技术委员会(SAC/TC4)提出并归口。

本文件起草单位:北京卓众出版有限公司、北京师范大学出版社(集团)有限公司、《中国科学》杂志社有限责任公司、中国科学院软件研究所、北京林业大学、上海大学、《中华医学杂志》社有限责任公司机械工业信息研究院、中国科学技术信息研究所。

本文件主要起草人:张品纯、陈浩元、任胜利、方梅、张铁明、刘志强、刘冰、梁福军、刘春燕。本文件及其所代替文件的历次版本发布情况为:

——1987年首次发布为GB/T7713-1987;

——本次为第一次修订。
\end{preamble}
\begin{preamble}{引言}
    无论是学术论文、学位论文还是科技报告,其撰写和编排都需要遵循一定的规范,以利于信息系统的收集、存储、处理、加工、检索、利用、交流、传播。GB/T 7713一1987《科学技术报告、学位论文和学术论文的编写格式》,对学术论文、学位论文和科技报告的撰写要求及编排格式作了统一规定。鉴于三者的使用对象及使用目的不尽相同,撰写要求及编排格式差异较大,后来修订 GB/T 7713 时,将其分为3个部分分别进行修订。

——第1部分:学位论文编写规则。目的在于规定了学位论文的撰写格式和要求。

——第2部分:学术论文编写规则。目的在于规定了学术论文的撰写要求和编排格式。

——第3部分:科技报告编写规则。目的在于规定了科技报告的编写、组织编排等要求。本文件描述了撰写和编排学术论文的基本要求和格式规范。学术论文编写的标准化和规范化,是使其格式和体例规范化,语言、文字和符号规范化,技术和计量单位标准化,以便于学术论文的检索和传播,促进学术成果的交流和使用。

本文件的适用范围,包括一切反映自然、社会和人文等的科学体系的学术论文。然而,由于学科门类、选定课题、研究工作方法、工作进行阶段、观测和调查等各方面的差异,采用本文件进行学术论文编写宜采取严肃性和灵活性相结合的原则。同时,人文社科类学术论文与科技类学术论文相比,具有内容表述丰富性和多样性等特征,人文社科类学术论文可在遵循本文件基本规定基础上,根据学科特点进一步制定具体的编写规范。

本文件对 GB/T7713-1987 中的学术论文编写内容进行了必要的检查、更新,进而形成单独的学术论文编写规则,代替GB/T7713一1987中的学术论文编写格式部分。
\end{preamble}

% mainmatter 是「正文」的意思。
% 从这里开始都是正文。
\mainmatter
\chapter{范围}

本文件规定了学术论文的组成部分以及撰写和编排的基本要求与格式。本文件适用于印刷版、缩微版、电子版等所有传播形式的学术论文。不同学科或领域的学术论文可参考本文件制定本学科或领域的编写规范。

\chapter{规范性引用文件}
下列文件中的内容通过文中的规范性引用而构成本文件必不可少的条款。其中,注日期的引用文件,仅该日期对应的版本适用于本文件;不注日期的引用文件,其最新版本(包括所有的修改单)适用于本文件。

GB 3100 国际单位制及其应用

GB/T 3101 有关量, 单位和符号的一般原则

GB/T 3102(所有部分) 量和单位

GB/T6447 文摘编写规则

GB/T 7408 数据元和交换格式 信息交换 日期和时间表示法

GB/T 7714 信息与文献 参考文献著录规则

GB/T 8170 数值修约规则与极限数值的表示和判定

GB/T 15834 标点符号用法

GB/T 15835 出版物上数字用法

GB/T 19996 公开版纸质地图质量评定

GB/T 28039 中国人名汉语拼音字母拼写规则

CY/T 35 科技书刊的章节编号方法

CY/T 119 学术出版规范 科学技术名词

CY/T 121 学术出版规范 注释

CY/T 170 学术出版规范 表格

CY/T 171 学术出版规范 插图

CY/T 173 学术出版规范 关键词编写规则

ISO 80000-1量和单位 第1部分:总则(Quantities and units-Part l:General)

ISO 80000-2量和单位第2部分:数学(Quantities and units-Part 2:Mathematics)


\chapter{术语和定义}
下列术语和定义适用于本文件。

\section{学术论文 academic paper}
对某个学科领域中的学术问题进行研究后,记录科学研究的过程、方法及结果,用于进行学术交流、讨论或出版发表, 或用作其他用途的书面材料。

注:在不引起混淆的情况下, 本文件中的“学术论文”简称为“论文”。

\section{正文部分mainbody}
论文的核心部分,通常由引言开始, 描述相关理论、实验(试验)、方法、假设和程序,陈述结果并进行讨论分析,阐明结论,以参考文献结尾。

\section{参考文献 reference}
对一个信息资源或其中一部分进行准确和详细著录的数据, 位于文末或文中的信息源。

[来源:GB/T7714-2015.3.1]


\chapter{组成部分}
\section{一般要求}
论文一般包括以下3个组成部分:

a) 前置部分;

b) 文部;

c) 附录部分。

论文各部分的构成及相关的元数据信息按照附录 A 进行。

\section{前置部分}
\subsection{题名}
题名是论文的总纲,是反映论文中重要特定内容的恰当、简明的词语的逻辑组合。题名中的词语应有助于选定关键词和编制题录、索引等二次文献所需的实用信息,应使用标准术语,学名全称药物和化学品通用名称,不应使用广义术语、夸张词语等。为便于交流和利用, 题名应简明,一般不宜超过 25 字。为利于国际交流,论文官有外文(多用英文)题名。

下列情况允许有副题名:题名语义未尽,用副题名补充说明论文中的特定内容:研究成果分几篇报道,或是分阶段的研究结果,各用不同副题名以区别其特定内容;其他有必要用副题名作为引申或说明者。

题名在论文中不同地方出现时应保持一致。

\subsection{作者信息}
论文应有作者信息。作者信息具有以下意义:拥有著作权的声明;文责自负的承诺;联系作者的渠道。作者信息的内容,一般包括作者姓名、工作单位及通信方式等。为利于国际交流, 论文宜有与中文对应的外文(多用英文)作者信息。

对论文有实际贡献的责任者应列为作者,包括参与选定研究课题和制订研究方案,直接参加全部或主要部分研究工作并作出相应贡献,以及参加论文撰写并能对内容负责的个人或单位。个人的研究成果,标注个人作者信息;集体的研究成果,标注集体作者信息即列出全部作者的姓名,不宜只列出课题组名称。标注集体作者信息时,应按对研究工作贡献的大小排列名次。

如需标注中国作者的汉语拼音姓名, 应执行 GB/T 28039 的规定, 即姓在前名在后, 双名连写, 其间不加短横线, 名不准许缩写。国外作者的姓名, 应尊重其各自的姓名拼写规则。

作者信息的位置宜置于题名之下。

论文可标注通信作者的有关信息。此项目也可标注在文末。

\subsection{摘要}
论文应有摘要。摘要是对论文的内容不加注释和评论的简短陈述,应具有独立性和自明性,即不阅读全文就可以获得必要的信息。为利于国际交流,宜有外文(多用英文)摘要。摘要的撰写应符合GB/T6447的规定。

摘要的内容通常包括研究的目的、方法、结果和结论。宜采用报道性摘要,也可采用报道/指示性摘要、指示性摘要。报道性摘要可采用结构式。

摘要中可以有数学式、化学式、插图、表格等,但不应含有数学式、化学式、插图、表格、参考文献等的编号, 不宜使用非公知公用的符号和术语。对摘要中首次出现非公知公用的简称、外文缩略语和缩写词, 应给出全称、中文翻译或解释。中文摘要的字数,原则上应与论文中的成果多少相适应, 在一般情况下, 报道性摘要以 400 字左右报道/指示性摘要以 300 字左右、指示性摘要以 150 字左右为宜。中文摘要、外文摘要内容宜对应,为利于国际交流, 外文摘要可以比中文摘要包含更多信息。

摘要宜置于作者信息之后。外文摘要可置于中文摘要之后,也可置于正文部分之后。

\subsection{关键词}
论文应有关键词。关键词是为便于文献检索从题名、摘要或正文部分选取出来用以表示论文主题内容的词或词组。关键词要有检索意义,不应使用太泛指的词,例如“方法”“理论”“分析”等。关键词的撰写应符合CY/T173 的规定。

关键词宜从《汉语主题词表》或专业词表中选取。未被词表收录的新学科、新技术中的重要术语以及地区、人物、产品等, 可选作关键词。

为利于国际交流, 宜标注与中文对应的外文(多用英文)关键词。

每篇论文以选取3个$\sim$8 个关键词为宜。

关键词宜置于摘要之后。

\subsection{其他项目}
论文前置部分要求、建议或允许标注的其他项目。

\begin{enumerate}
\item 基金资助项目产出的论文,应标注该基金名称及项目编号。

\item 宜标注收稿日期,可同时标注修回日期。此项目也可标注在文末。

\item  可标注引用本论文的参考文献格式。

\item  可标注论文增强出版的元素以及相关声明,如二维码,网址链接、作者声明等。此类元素也可标注在论文其他部分的适当处。
\end{enumerate}
\section{正文部分}
\subsection{一般要求}
正文部分通常包括引言、主体、结论和参考文献等。正文的表述应科学合理、客观真实、准确完整、层次清晰、逻辑严密、文字顺畅。

\subsection{引言}
学术论文一般有引言。引言内容通常包含研究的背景、目的、理由、预期结果及其意义和价值。

引言的编写宜做到切合主题, 言简意赅, 突出重点、创新点,客观评介前人的研究,如实介绍作者自己的成果。

\subsection{主体}
主体部分是论文的核心,占论文的主要篇幅,论文的论点、论据和论证均在此部分阐述或展示。

主体部分应完整描述研究工作的理论、方法、假设、技术、工艺、程序、参数选择等,清晰说明使用的关键设备装置、仪器仪表、材料原料,或者涉及的研究对象等,以便于本专业领域的读者可依据这些描述重复研究过程;应详细陈述研究工作的过程、步骤及结果,提供必要的插图、表格、计算公式、数据资料等信息,并对其进行适当的说明和讨论。

主体部分的结构,一般由具有逻辑关系的多章构成,如理论分析、材料与方法、结果和讨论等内容,均宜独立成章。

\subsection{结论}
结论是对研究结果和论点的提炼与概括,不是摘要或主体部分中各章、节小结的简单重复,宜做到客观、准确、精练、完整。结论应编章编号。

如果推导不出结论,也可没有“结论”而写作“结束语”,进行必要的讨论,在讨论中提出建议或待研究解决的问题等。

\subsection{致谢}
致谢是作者对论文的生成作过贡献的组织或个人予以感谢的文字记录,内容应客观、真实,语言宜诚恳、真挚、恰当。

致谢内容可用与正文部分相区别的字体,排在结论或结束语之后,一般不编章编号。

\subsection{参考文献}
论文中应引用与研究主题密切相关的参考文献。

参考文献的著录项目、著录符号、著录格式以及参考文献在正文中的标注法,应符合 GB/T 7714的规定。

参考文献表既可采用顺序编码制,也可采用著者-出版年制,但全文应统一。采用顺序编码制组织的参考文献表应置于文末,也可用脚注方式将参考文献置于当页地脚处。

列于文末的参考文献表可以编章编号。

\section{附录部分}
附录部分是以附录的形式对正文部分的有关内容进行补充说明。

论文一般不设附录;但那些编入正文部分会影响编排的条理性和逻辑性、有碍论文结构的紧凑性对突出主题有较大价值的材料,以及某些重要的原始数据、数学推导、计算程序、设备、技术等的详细描述, 可作为附录编排于论文的末尾。


\chapter{编排格式}
\section{一般要求}
论文应遵守《中华人民共和国国家通用语言文字法》,采用国务院发布的《通用规范汉字表》的规范汉字编写,遣词造句应符合汉语语法,标点符号使用应符合 GB/T 15834 的规定,文字表达做到题文相符、结构严谨、符合逻辑、用词准确、语言通顺。

论文涉及的编号、量和单位、插图、表格、数字、数学式,注释、科学技术名词等的表达,均应符合规范性引用文件的规定。

印刷版论文宜用 A4 幅面纸张。用纸、用墨、版面设计等应便于论文的印刷、装订、阅读、复制和缩微。

电子版论文应采用通用文件格式,并可提供音频、视频、数据集等数字化资料。

论文中各部分文字的字号和字体见附录 B。

\section{编号}
\subsection{ 一般要求}
为使论文条理清晰,易于辨认和引用,章、节、条、款、项、段以及插图、表格、数学式等的编号方法符合CY/T35的规定。

论文如有需要,也可采用传统的编号方法。

\subsection{章节编号}
正文部分应根据需要划分章节,一般不宜超过 4级。章应有标题, 节宜有标题,但在某一章或节中,同一层次的节,有无标题应统一。章节标题一般不宜超过15字。

章节的编号宜采用阿拉伯数字。不同层次竟节数字之间用下圆点相隔。末位数字后不加占号,如引言编号“0”;章编号“1”“2”...;节编号“2.1”“2.2”....., “3.2.1"“3.2.2”...。各层次章节编号全部顶格排,其后空1个汉字的间隙接排标题,标题末尾不加标点,正文另起行。

章节的编号如选择传统方法,可混合使用汉字数字和阿拉伯数字。

注:如果引言部分不用“引言”二字,则不编章编号“0”。

\subsection{列项说明编号}
列项说明指论文的某些内容需要分条或分款来说明的一类表述形式。

列项说明时,宜在各项前添加采用阿拉伯数字或小写拉丁字的编号,如:“1)” “2)”,“(1)” “(2)”, “a)”\linebreak “b)”,“(a)”“(b)”。如果论文中已经把形式为“(1)”“(2)”的编号作为数学式的序号,则不宜将其用于列项说明。列项说明的各项前,也可采用符号,如“---”“。”等。

\subsection{ 插图表格、数学式编号}
插图、表格、数学式等一律用阿拉伯数字分别依序连续编号。

一般按出现先后顺序全文统一编号,如“图1”“图2”“表1”“表2”“式(1)”“式(2)”等。

只有1幅插图,1个表格时,应编为“图1”“表1”。

\subsection{附录编号}
论文如有附录,采用大写拉丁字母依序连续编号,如附录A、附录B等。

\section{量和单位}
\SubSectionStart
\SubSection{论文中使用量和单位的名称、符号、书写规则都应符合 GB 3100、GB/T 3101、GB/T 3102(所有部分)的规定。}
\SubSection{应采用标准化的量名称,不应使用已废弃的量名称(如“电流强度”“定压质量热容”“体积百分浓度”应分别为“电流”“质量定压热容”“体积分数”)和用“单位+数”构成的量名称(如“克数”“天数”“摩尔数”应分别为“质量”“时间”“物质的量”)。}
\SubSection{应采用标准化的量符号。量符号通常为单个拉丁字母或希腊字母,描述传递现象的特征数由 2 个字母组成, 并一律用斜体(pH 除外)。为区别不同的使用情况, 可按有关规定在量符号上附加下标或其他的说明性标记.并注意区分量的下标字母的正斜体、大小写。}

%——构造图、装配图中的尺寸数据如具有相同的单位,宜将共同单位标注在图的右下角或左下角写作“单位:XX”。
%
%——地图插图应确保准确无误, 应符合 GB/T19996 的规定。
\SubSection{应使用法定计量单位,不使用已废弃的非法定计量单位。个别科技领域如有特殊需要,且相关学科国际组织的规范中也允许使用,则可使用某些非法定计量单位,如可用bar(巴)、var(乏)、\AA(埃)、Ci(居里)、mmHg(毫米汞柱)等。}
\SubSection{在插图、表格、数学式和文字叙述中,表达量值时,一律使用单位的国际符号,且无例外地用正体字母。单位符号与其前面的数值之间应留适当空隙,如 20 $^\circ$C、1.84 g/mL不应写作 20 $^\circ$C、1.84g/mL。}
\noindent 不准许对单位符号进行修饰,如添加上下标,或在组合单位符号中插入化学元素符号等说明性记号。
\SubSection{
不应把单位英文名称的缩写(如rpm、kmph、bps)和表示数量份额的缩写(如 ppm、pphm、ppb、ppt)作为单位符号使用。对 ppm 等缩写,宜采用 10 的乘方形式替代。}
\SubSection{宜使用国际单位制(SI)词头构成十进倍数或分数单位,并应符合相关规则:}
\begin{enumerate}
\item 词头不准许独立使用,如$\mu$m不应写作$\mu$;
\item 词头不准许重叠使用,如GHz不应写作kMHz;
\item 平面角单位°、\verb|'|、\verb|"|和时间单位d、h、min等不准许用SI词头构成倍数或分数单位;摄氏温度单位$^\circ$C前允许加词头,如k$^\circ$C;
\item 词头符号与所紧接的非组合单位的符号应作为一个整体对待,并具有相同的幂次,
如:$1\mu s^{-1}=(10^{-6}s)^{-1}=10^6s^{-1}$。
\end{enumerate}
\SubSection{正确书写二进制倍数词头。依据ISO 80000-1,8个二进制倍数词头符号应分别为:Ki($2^{10}$),Mi($2^ {20}$),Gi(($2^ {30}$),Ti($2^ {40}$),Pi($2^ {50}$),Ei($2 ^{60}$),Zi($2 ^{70}$),Yi($2^{80}$)。}
\SubSection{量和单位的使用还应注意以下问题:}
\begin{enumerate}
\item 量值相乘表示面积、体积等时,每个量的单位应重复写出,如40 m$\times$60 m不应写作40$\times$60 m或40$\times$60 m$^{2}$;
\item 单位相同的量值范围,前一个量的单位宜省略,如1.5$\sim$3.6 mA不必写作1.5 mA$\sim$3.6 mA,但20\%$\sim$30\%等例外,前一个量的单位不应省略;
\item 单位相同的一组量值中,可只保留最末一个量值的单位,如15、20、25 $^\circ$C;
\item “\%”“\textperthousand ”是1的分数单位符号,“\%”可用来替代0.01或10$^{-2}$,“\textperthousand ”可用来替代0.001或10$^{-3}$。
\end{enumerate}
\SubSectionEnd
\section{插图}
\SubSectionStart
\SubSection{插图是论文重要的组成部分,包括坐标曲线图、构造图、示意图、框图、流程图、记录图、地图、照片等。插图应具有自明性、简明性、科学性和艺术性,大小适当,图中文字清晰可见,其编排应符合CY/T 171的规定。}
\SubSection{插图应有编号,编号方法见5.2.4。}
\SubSection{插图应有图题,置于图编号之后,并空1个汉字的间隙。图编号与图题应居中置于图的下方。必要时,可有简明的图例、图注或说明。图注或说明为多条并需编序号时,宜采用阿拉伯数字加后半圆括号或圈码,置于被注对象的右上角,如xxxx$^{2)}$或xxxx$^{\text{\ding{173}}}$。图注或说明的末尾应加“。”。}
\SubSection{不同类型的插图有不同的编排要求,编排时应符合下列要求}

\begin{itemize}[nosep, left=2\ccwd,label={——}]
\item 坐标曲线图的标目应分别置于横、纵坐标轴的外侧,一般居中排。横坐标标目应自左至右;纵坐标标目应自下而上,“顶左底右”;如有右侧纵坐标,其标目排法同左侧。当标目同时用量和单位表示时,应采用“量的符号或名称/单位符号”的标准化形式,如cB/(mol/L)、B的浓度/(mol/L)、BMI/(kg/m$^2$)(BMI为身体质量指数的缩写词)。

\item 照片图的主题和主要显示部分应轮廓鲜明。如采用放大或缩小的复制品,应图像清晰、反差适中。照片上应有表示目的物尺寸的标度。

\item 构造图、装配图中的尺寸数据如具有相同的单位,宜将共同单位标注在图的右下角或左下角,写作“单位:XX”。

\item 地图插图应确保准确无误,应符合GB/T 19996的规定。
\end{itemize}
\SubSection{插图宜紧置于首次提及该图编号的正文之后, 先见文字后见图。由几个分图组成的插图如需转页接排,可在所有分图都排完之后排图编号、图题。}
\SubSectionEnd
\section{表格}
\SubSectionStart
\SubSection{ 表格是论文重要的组成部分,应具有自明性、简明性、规范性和逻辑性,其编排应符合 CY/T 170的规定。}
\SubSection{表格应有编号, 编号方法见 5.2.4。}
\SubSection{表格应有表题,置于表编号之后,并空 1个汉字的空隙。表编号和表题应置于表格顶线上方,宜居中排。必要时,可将表中的符号、标记、代码及需要说明的事项,用简练的文字,作为表注置于表的下方。表注为多条并需编序号时,宜采用阿拉伯数字加后半圆括号或圈码,置于被注对象的右上角,如XXXX”或XXXX。表注的末尾应加”。}
\SubSection{表格应有表头,表头中不准许使用斜线。表格的编排,宜将内容和测试项目由左至右横排,数据依序竖排。
表头栏目的标注应正确、齐全。表格中内容相同的相邻栏或上下栏,应重复写出,或以通栏表示,不应用“同左”“同上"等字样代替。表身中的“空白”表示无此项或未测量,“—”表示测量过而未发现,“0”表示实测结果为零。} 

\textbf{注:}当“—”可能与代表阴性相混时, 可用“...”

当表格中某一栏目同时用量和单位表示时, 应采用“量的符号或名称/单位符号”的标准化形式,如c${}_p$/[J/(kg$\cdot$K)]、质量定压热容/[J/(kg$\cdot$K)]、CHT/kK(CHT 为临界高温的缩写词)。若全表格所有栏目的单位都相同, 宜将共同单位标注在表格的右上方。

\SubSection{表格宜紧置于首次提及该表编号的正文之后, 先见文字后见表。如果某个表格需要转页接排则应在随后接排该表的表格上方加“表X(续)”或“续表”字样。续表应重复表头。}

\SubSectionEnd
\section{数字}
\SubSectionStart
\SubSection{数字用法应符合 GB/T 15835 的有关规定。鉴于阿拉伯数字具有笔画简单、结构科学、形象清晰、组数简短、国际通用等优点及科技语言的特殊性,论文中数字使用的总原则是:凡是可以使用阿拉伯数字, 而且又很简明清晰的地方,宜使用阿拉伯数字。}
\SubSection{为达到醒目、易于辨识的效果, 下列场合应使用阿拉伯数字:}
\begin{enumerate}
\item  计量和计数的数字, 如应写作 20 kg、35 m/s、30$\sim$40 mL、365、15.8\%、2/3、4人等;

\item 编号的数字, 如应写作010-62736603、104 国道、国发[2020]8号文件等;

\item  表示公历世纪、年代、年份、日期和时刻的数字,应符合GB/T 7408和 GB/T 15835的相关规定,如应写作20世纪50-70年代、2016-2020年、2020年8月28日9时38分5秒(也可采用全数字表示法写作2020-08-28T09:38:05)等;
\item 已定型名称中的数字,如应写作5G 手机PM$_{2.5}$质量浓度维生素 B$_{12}$、97号汽油“3$\cdot$15”消费者权益日等。
\end{enumerate}
\SubSection{科学计量中的数值修约和极限数值的表示和判定,应符合GB/T 8170给出的规则。连续性数据分组时,每组数据的量值范围应准确表示, 如长度0$\sim$20 m 平均分为4 组应写作$0\sim<5$m、$5\sim<10$ m、$10\sim<15$ m、$15\sim20$ m, 也可写作$[0,  5)$m、$[5,  10)$m、$[10,  15)$m、 $[15,  20]$m,但不应写作$0\sim5$m、$5\sim10$m、$10\sim15$m、$15\sim20$m。}
\SubSection{阿拉伯数字的使用还应注意以下规范:}
\begin{enumerate}
  \item  大于999的整数和多于3位数的小数,均宜采用三位分节法分节,即从小数点起向左或向右每3位留适当空隙,如写作1 000、0.000 1;
  \item  数值的有效数字应全部写出,如“1.50,1.75,2.00”不应写作“1.5,1.75,2”;
  \item  阿拉伯数字不准许与除“万”“亿”和SI词头中文符号以外的数词连用,如3 500元不应写作3千5百元,我国2020年人口普查人数1 411 778 724人可写作14亿1177万8724人;
  \item  有起点和终点的时间段之间应采用一字线连接,如2020-09-01—12-01不应写作2020-09-01$\sim$12-01。
\end{enumerate}
\SubSection{下列场合应使用汉字数字:}
\begin{enumerate}
\item  作为词素构成定型的词、词组、惯用语、缩略语等的数字, 如二倍体、三叶虫、二元三次方程、四氧化三铁、十二指肠、五行、五运六气、三焦、“十四五”规划等;

\item  2个数字连用表示的概数和“几”字前后的数字, 如三五天、五六小时, 七八十米、三十几摄氏度、几十吨等;

\item 非公历纪年的数字, 如清咸丰十年九月二十日(1860 年 11月2日)民国二十七年(1938 年)。
\end{enumerate}
\SubSectionEnd
\section{ 数学式}
\SubSectionStart
\SubSection{数学式中的变量、变动的附标、函数、有定义的已知函数、其值不变的数学常数、已定义的算子、特殊集合符号、矢量或向量、矩阵以及说明性的字符等,编排时使用的大小写、正斜体、黑白体等,均应符合GB/T 3102.11的规定。}
\SubSection{ 注意区分与单位无关的量关系式和与单位有关的数值关系式, 二者之间宜首选前者。数学式应以正确的数学形式表示, 由字母符号表示的变量, 应随数学式对其含义进行解释。示例1和示例2分别为量关系式和数值关系式的式样。}

\textbf{\heiti 示例1:}
\[v=l/t\]
式中:$v$ 为匀速运动质点的速度,$l$ 为运行距离,$t$ 为时间间隔。

\textbf{\heiti 示例2:}
\[v=3.6l/t\]
式中:$v$ 为匀速运动质点的速度的数值,单位km/h;$l$ 为运行距离的数值,单位m;$t$ 为时间间隔的数值,单位s。

注:在一篇论文中,同一个符号不应既表示一个物理量,又表示其对应的数值。
\SubSection{数学式不应使用量的名称或描述量的术语表示。量的名称或多字缩略术语, 不论正体或斜体, 亦不论是否含有下标, 都不应该用来代替量的符号。}

\textbf{\heiti 正确}
$$
t_i=\sqrt{\frac{S_{\mathrm{ME},  i}}{S_{\mathrm{MR},  i}}}
$$

式中: $t_i$ 为系统 $i$ 的统计量,  $S_{\mathrm{ME},  i}$ 为系统 $i$ 的残差均方,  $S_{\mathrm{MR},  i}$ 为系统 $i$ 由于回归产生的均方。

\textbf{\heiti 不正确}
$$
t_i=\sqrt{\frac{M S E_i}{M S R_i}}
$$

式中: $t_i$ 为系统 $i$ 的统计量,$M S E_i$ 为系统 $i$ 的残差均方,  $M S R_i$ 为系统 $i$ 由于回归产生的均方。
\SubSection{数学式一般串文排,下文要提及的编有式编号的公式,大公式(如繁分式, 积分式、连乘式、求和式、矩阵、行列式等),应另行居中排,式编号标注于该式所在行(或转行式的末行)的最右端。居中排数学式的结尾, 允许按其在行文中的语法关系添加标点符号。}


依据GB/T 3102.11,数学式需要断开转行排的首选规则为:在$=$、$\approx$、$<$、$>$、$\ne$、$>$等关系符号或$+$、$-$、$\pm$、$\mp$、$\times$、$\cdot$、$\div$、/等运算符号后断开,而在下一行开头不应重复这一符号。


\textbf{\heiti 示例 1:}

\begin{align*}
&W\left(N_{1}\right)=H_{0, 1}+\int_{\tau^{-1}}^{\tau^{-1}+1} L_{z} \mathrm{e}^{-2 \min N_{1}} \mathrm{~d} \alpha= \\
&R\left(N_{0}\right)+\int_{\tau^{-1}}^{-\tau^{-1}+1} L_{\alpha}^{r} \mathrm{e}^{-2 \min \mathrm{N}_{1}} \mathrm{~d} a+O\left(P^{r-\alpha-v}\right) \\
\end{align*}


按照 ISO 80000-2,  数学式也可在  $=$、$ \approx$ 、 $\neq$ 、 $\leqslant$  等关系符号和  $+$、$-$ 、$\times$、$/$等运算符号前断开, 上一行末尾不重复这一符号。


\textbf{\heiti 示例 2:}

\begin{align*}
f(x,  y)= & f(0, 0)+\frac{1}{1 !}\left(x \frac{\partial}{\partial x}+y \frac{\partial}{\partial y}\right) f(0, 0) \\
& +\frac{1}{2 !}\left(x \frac{\partial}{\partial x}+y \frac{\partial}{\partial y}\right)^{2} f(0, 0)+\cdots \\
& +\frac{1}{n !}\left(x \frac{\partial}{\partial x}+y \frac{\partial}{\partial y}\right)^{n} f(0, 0)+\cdots
\end{align*}

\SubSection{关于数学式表示的建议:}
\begin{enumerate}
\item  在行文中宜避免使用多于 1行的表示形式, 如 m/V 优于$\dfrac{\rm m}{\rm V}$;

\item  在数学式中宜避免使用多于1个层次的上标或下标符号 \verb|,  | 如 P${}_{1, \min{}}$优于 P${}_{1_{\min{}}}$;

\item  在数学式中宜避免使用多于 2行的表示形式。
\end{enumerate}

\textbf{\heiti 示例:}

使用
\[
\frac{\sin \left[(N+1)_{\alpha} / 2\right] \sin \left(N_{\alpha} / 2\right)}{\sin (\alpha / 2)}=\cdots \ldots
\]

不使用
\[
\frac{\sin \left[\frac{(N+1)}{2} \alpha\right] \sin \left(\frac{N}{2} \alpha\right)}{\sin \frac{\alpha}{2}}=\cdots \cdots
\]

\SubSectionEnd
\section{注释}
除图注、表注及参考文献的地脚注外,论文中的文字内容需要加以说明又不适于作正文来叙述时可采用注释。

注释的标注应符合 CY/T 121 的规定。宜用文中编号加脚注的方式,置于所注释正文所在页的底部。注释编号应与参考文献脚注的圈码相区别。

\section{科学技术名词}
科学技术名词简称科技名词,也称术语,其使用应符合 CY/T 119 的如下规定。

\begin{enumerate}
 \item  科学技术名词应首选全国科学技术名词审定委员会审定公布的规范名词。“全称”和“简称”均可使用,减少使用“又称”,不宜使用“俗称”或“曾称”。

\item 不同机构公布的规范名词不一致时,可选择使用。同一机构对同一概念的定名在不同学科或专业领域不一致时,宜依论文所在学科或专业领域选择使用规范名词。

\item 尚未审定公布的科学技术名词,宜使用单义性强、贴近科学内涵或行业习惯的名词

\item 尽量少用字母词。如果使用未经审定公布的字母词,应在首次出现时括注其中文译名,必要时还应同时括注其外文全称。

\item 同一篇论文使用的科学技术名词应保持前后一致
\end{enumerate}%
%\chapter{模板的基本使用}
%「要求」中大多数细节都能通过 \LaTeX{} 比较舒服地实现,唯独「索引」\index{索引}部分是个老大难。好在经过一番实践,最终是将之搞定,不过也因此给模板的使用带来了一些问题——主要是需要进行一些安装和配置。
%
%除掉模板文件内包含的东西,还需要额外下载安装名为 Python\index{Python} 的脚本语言解释器。你可以在\href{https://www.python.org/ftp/python/2.7.6/}{\fbox{这里}}找到适用于全部平台的 Python。比如适用于 32 位 Windows 系统的版本是\href{https://www.python.org/ftp/python/2.7.6/python-2.7.6.msi}{\fbox{这个}},而 64 位的 Windows 系统则应该安装\href{https://www.python.org/ftp/python/2.7.6/python-2.7.6.amd64.msi}{\fbox{这个}}。
%
%此外,为了让 \LaTeX{} 能够载入外部程序,在执行 \hologo{XeLaTeX} 的时候,还必须给它加上 \verb|--shell-escape| \index{-{}-shell-escape}选项,所以整个命令看起来像是下面这样:
%
%\begin{quote}
%\begin{verbatim}
%xelatex --shell-escape GB-template
%bibtex GB-template
%xelatex --shell-escape GB-template
%xelatex --shell-escape GB-template
%\end{verbatim}
%\end{quote}
%
%当然,这有点复杂了。所以我准备了名为 \verb|make.bat| 的批处理文件,自动做这些工作。你只需要在编写好 \LaTeX{} 文档之后,双击运行这个文件就会生成 PDF 了。
%
%最后,为了保证文章内没有乱码,所有的文件必须使用 UTF-8 编码\index{编码}来保存。这一点尤为重要,但却不是我所能控制的了,因为这是「用户」使用时候应当注意的问题。

%\input{test.tex}
% appendix 开始是「附件」
\appendix\autoclearpage
\gchapter{学术论文的构成元素}

表A.1规定了学术论文的构成元素

\begin{table}[!htbp]
\centering
\caption{学术论文的构成元素}
    \begin{tabularx}{\linewidth}{|p{2.3cm}|p{3cm}|p{3cm}|X|}
    \hline
 \multicolumn{2}{|c|}{  组成 }         & \multicolumn{1}{c|}{必备性}   & \multicolumn{1}{c|}{功能} \\
    \hline
      & 题名    & 必备    & 提供题名元数据信息 \\
    \cline{2-4}
          & 作者信息  & 必备    & 提供作者元数据信息 \\
   \cline{2-4}
   \multicolumn{1}{|c|}{前置部分}        & 摘要    & 必备    & 提供摘要元数据信息 \\
    \cline{2-4}
          & 关键词   & 必备    & 提供关键词元数据信息 \\
\cline{2-4}
          & 其他项目  & 部分必备或可选 & 提供管理与利用元数据信息 \\
    \hline
    & 引言    & 必备    & 内容 \\
   \cline{2-4}
          & 主体    & 必备    & 内容 \\
 \cline{2-4}
     \multicolumn{1}{|c|}{正文部分}      & 结论    & 有则必备  & 内容 \\
  \cline{2-4}
          & 致谢    & 可选    & 内容 \\
\cline{2-4}
          & 参考文献  & 必备    & 结构元数据 \\
    \hline
    附录部分  & 附录    & 有则必备  & 结构元数据 \\
    \hline
    \end{tabularx}%
\end{table}
\newpage
\zchapter{学术论文中使用的字号和字体}

学术论文编写中各部分文字使用的字号和字体可参考表B.1。

\begin{table}[!htbp]
\centering
\caption{学术论文中使用的字号和字体}
        \begin{tabularx}{\linewidth}{|l|l|X|}
     \hline
    {组成部分} & {文字内容} & {字号和字体} \\
    \hline
    \multirow{6}[18]{*}{{前置部分}} & {中文题名} & {小2号黑体} \\
\cline{2-3}     & {作者姓名} & {小4号楷体} \\
\cline{2-3}     & {工作单位及通信方式} & {小5号宋体} \\
\cline{2-3}     & {中文摘要、关键词} & {引题小5号黑体,内容小5号仿宋} \\
\cline{2-3}     & {英文题名} & {4号黑体} \\
\cline{2-3}     & {英文作者姓名} & {5号宋体} \\
\cline{2-3}     & {英文工作单位及通信方式} & {小5号宋体} \\
\cline{2-3}     & {英文摘要、关键词} & {引题小5号黑体、内容小5号宋体} \\
\cline{2-3}     & {其他项目} & {小5号宋体} \\
    \hline
    \multirow{5}[14]{*}{{正文部分}} & {引言、主体、结论的章编号和标题} & {小4号黑体} \\
\cline{2-3}     & {引育、主体、结论的节编号和标题} & {5号黑体} \\
\cline{2-3}     &  {引言、主体、结论的正文内容} & {5号宋体} \\
\cline{2-3}     &  {插图、表格编号和标题} & {小5号黑体} \\
\cline{2-3}     &  {表格内容、表注和图注} & {小5号宋体} \\
\cline{2-3}     & {致谢} & {引题5号黑体,内容5号楷体} \\
\cline{2-3}     & {参考文献} & {引题(及章编号)小4号黑体,内容小5号宋体} \\
    \hline
    {附录部分} & {附录} & {编号、标题小4号黑体,内容5号宋体} \\
    \hline
    \end{tabularx}%
\end{table}% 


\newpage
\begin{thebibliography}{9}

\bibitem{1} GB/T 3179一2009 期刊编排格式
\bibitem{1} 中国科学技术信息研究所,北京图书馆.汉语主题词表:工程技术卷:第1-13册[M].北京:科学技术文献出版社,2014.

\bibitem{1} 中国科学技术信息研究所,北京图书馆.汉语主题词表:自然科学卷:第1-5册[M].北京:科学技术文献出版社,2018.

\bibitem{1} 国际计量局.国际单位制(SI)[M].7版.北京:科学出版社,2000.

\bibitem{1} 中华人民共和国国家通用语言文字法(中华人民共和国主席令第37号)

\bibitem{1} 国务院关于公布《通用规范汉字表》的通知(国发[2013]23号)

\end{thebibliography}
\end{document}
