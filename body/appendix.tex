\gchapter{学术论文的构成元素}

表A.1规定了学术论文的构成元素

\begin{table}[!htbp]
\centering
\caption{学术论文的构成元素}
    \begin{tabularx}{\linewidth}{|p{2.3cm}|p{3cm}|p{3cm}|X|}
    \hline
 \multicolumn{2}{|c|}{  组成 }         & \multicolumn{1}{c|}{必备性}   & \multicolumn{1}{c|}{功能} \\
    \hline
      & 题名    & 必备    & 提供题名元数据信息 \\
    \cline{2-4}
          & 作者信息  & 必备    & 提供作者元数据信息 \\
   \cline{2-4}
   \multicolumn{1}{|c|}{前置部分}        & 摘要    & 必备    & 提供摘要元数据信息 \\
    \cline{2-4}
          & 关键词   & 必备    & 提供关键词元数据信息 \\
\cline{2-4}
          & 其他项目  & 部分必备或可选 & 提供管理与利用元数据信息 \\
    \hline
    & 引言    & 必备    & 内容 \\
   \cline{2-4}
          & 主体    & 必备    & 内容 \\
 \cline{2-4}
     \multicolumn{1}{|c|}{正文部分}      & 结论    & 有则必备  & 内容 \\
  \cline{2-4}
          & 致谢    & 可选    & 内容 \\
\cline{2-4}
          & 参考文献  & 必备    & 结构元数据 \\
    \hline
    附录部分  & 附录    & 有则必备  & 结构元数据 \\
    \hline
    \end{tabularx}%
\end{table}
\newpage
\zchapter{学术论文中使用的字号和字体}

学术论文编写中各部分文字使用的字号和字体可参考表B.1。

\begin{table}[!htbp]
\centering
\caption{学术论文中使用的字号和字体}
        \begin{tabularx}{\linewidth}{|l|l|X|}
     \hline
    {组成部分} & {文字内容} & {字号和字体} \\
    \hline
    \multirow{6}[18]{*}{{前置部分}} & {中文题名} & {小2号黑体} \\
\cline{2-3}     & {作者姓名} & {小4号楷体} \\
\cline{2-3}     & {工作单位及通信方式} & {小5号宋体} \\
\cline{2-3}     & {中文摘要、关键词} & {引题小5号黑体,内容小5号仿宋} \\
\cline{2-3}     & {英文题名} & {4号黑体} \\
\cline{2-3}     & {英文作者姓名} & {5号宋体} \\
\cline{2-3}     & {英文工作单位及通信方式} & {小5号宋体} \\
\cline{2-3}     & {英文摘要、关键词} & {引题小5号黑体、内容小5号宋体} \\
\cline{2-3}     & {其他项目} & {小5号宋体} \\
    \hline
    \multirow{5}[14]{*}{{正文部分}} & {引言、主体、结论的章编号和标题} & {小4号黑体} \\
\cline{2-3}     & {引育、主体、结论的节编号和标题} & {5号黑体} \\
\cline{2-3}     &  {引言、主体、结论的正文内容} & {5号宋体} \\
\cline{2-3}     &  {插图、表格编号和标题} & {小5号黑体} \\
\cline{2-3}     &  {表格内容、表注和图注} & {小5号宋体} \\
\cline{2-3}     & {致谢} & {引题5号黑体,内容5号楷体} \\
\cline{2-3}     & {参考文献} & {引题(及章编号)小4号黑体,内容小5号宋体} \\
    \hline
    {附录部分} & {附录} & {编号、标题小4号黑体,内容5号宋体} \\
    \hline
    \end{tabularx}%
\end{table}% 