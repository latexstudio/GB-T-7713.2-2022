\begin{preamble}{重排自序}
\begin{mybox}{重排说明 }{gray}

最近不少用户找我要这个标准《学术论文编写规则》的可编辑版本,在网上找到的都是图片,我就特此制作了这个版本。内容校对了几次,可能还有疏漏。仅供参考。

如果有问题,可以加我 微信:t314159265 , 感谢各位!

大家还是查看官网的版本,地址:\url{https://std.samr.gov.cn/gb/search/gbDetailed?id=F159DFC2A91247EFE05397BE0A0AF334}

\end{mybox}
\end{preamble}

\begin{preamble}{前言}
  本文件按照GB/T1.1-2020《标准化工作导则 第1部分:标准化文件的结构和起草规则》的规定起草。

本文件是GB/T7713的第2部分。GB/T7713已经发布了以下部分:

——第1部分:学位论文编写规则:

——第2部分:学术论文编写规则:

——第3部分:科技报告编写规则。

本文件部分代替GB/T7713-1987《科学技术报告、学位论文和学术论文的编写格式》,与GB/T7713一1987相比,除结构调整和编辑性改动外,主要技术变化如下:

\begin{enumerate}[label={\alph*)}]
  \item
 将其适用范围扩展至印刷版、缩微版、电子版等所有传播形式的学术论文(见第 1章);

  \item   将“引言”更改为“范围”,更改了相关表述(见第1章,1987 年版的第1章);

  \item   将“定义”更改为“术语和定义”,删除了与学术论文编写规则无关的术语和定义,更改了“学术论文”的定义,增加了“正文部分”“参考文献”的定义(见第3章,1987 年版的第2章);

  \item  将“编写要求”“编写格式”更改为“编排格式”(见第5章,1987 年版的第3章第4章);

  \item  将“前置部分”“主体部分”“附录”“结尾部分”更改为“组成部分”“编排格式”(见第 4 章第5章,1987年版的第5章、第6章、第7章、第8章);

  \item  “组成部分”的更改:关于题名字数,将“题名一般不宜超过 20 字”更改为“为便于交流和利用题名应简明,一般不宜超过 25字”;关于摘要,将“中文摘要一般不宜超过 200$\sim$300字;外文摘要不宜超过 250 个实词”更改为“中文摘要的字数,原则上应与论文中的成果多少相适应,在般情况下,报道性摘要以400字左右报道/指示性摘要以300 字左右指示性摘要以150字左右为宜。中文摘要、外文摘要内容宜对应,为利于国际交流,外文摘要可以比中文摘要包含更多信息”,删除了“除了实在无变通办法可用以外,摘要中不用图、表、化学结构式”;在“其他项目”中,增加了学术论文前置部分要求、建议或允许标注的项目,如基金名称及项目编号、收稿日期、引用本论文的参考文献格式,论文增强出版的元素以及相关声明等(见4.2.1、4.2.3、4.2.5,1987年版的5.5.1、5.7.4、5.7.5);

  \item  在“编排格式”中,选列了学术论文的编号、量和单位、插图、表格数字、数学式、注释、科学技术名词的规范化要点及示例(见5.2、5.3、5.4、5.5、5.6、5.7、5.8、5.9);

  \item   删除了附录 A“封面示例”和附录 B“相关标准”,增加了规范性附录 A“学术论文的构成要素”(见附录A.1987年版的附录A附录B)。

\end{enumerate}

请注意本文件的某些内容可能涉及专利。本文件的发布机构不承担识别专利的责任。

本文件由全国信息与文献标准化技术委员会(SAC/TC4)提出并归口。

本文件起草单位:北京卓众出版有限公司、北京师范大学出版社(集团)有限公司、《中国科学》杂志社有限责任公司、中国科学院软件研究所、北京林业大学、上海大学、《中华医学杂志》社有限责任公司机械工业信息研究院、中国科学技术信息研究所。

本文件主要起草人:张品纯、陈浩元、任胜利、方梅、张铁明、刘志强、刘冰、梁福军、刘春燕。本文件及其所代替文件的历次版本发布情况为:

——1987年首次发布为GB/T7713-1987;

——本次为第一次修订。
\end{preamble}
\begin{preamble}{引言}
    无论是学术论文、学位论文还是科技报告,其撰写和编排都需要遵循一定的规范,以利于信息系统的收集、存储、处理、加工、检索、利用、交流、传播。GB/T 7713一1987《科学技术报告、学位论文和学术论文的编写格式》,对学术论文、学位论文和科技报告的撰写要求及编排格式作了统一规定。鉴于三者的使用对象及使用目的不尽相同,撰写要求及编排格式差异较大,后来修订 GB/T 7713 时,将其分为3个部分分别进行修订。

——第1部分:学位论文编写规则。目的在于规定了学位论文的撰写格式和要求。

——第2部分:学术论文编写规则。目的在于规定了学术论文的撰写要求和编排格式。

——第3部分:科技报告编写规则。目的在于规定了科技报告的编写、组织编排等要求。本文件描述了撰写和编排学术论文的基本要求和格式规范。学术论文编写的标准化和规范化,是使其格式和体例规范化,语言、文字和符号规范化,技术和计量单位标准化,以便于学术论文的检索和传播,促进学术成果的交流和使用。

本文件的适用范围,包括一切反映自然、社会和人文等的科学体系的学术论文。然而,由于学科门类、选定课题、研究工作方法、工作进行阶段、观测和调查等各方面的差异,采用本文件进行学术论文编写宜采取严肃性和灵活性相结合的原则。同时,人文社科类学术论文与科技类学术论文相比,具有内容表述丰富性和多样性等特征,人文社科类学术论文可在遵循本文件基本规定基础上,根据学科特点进一步制定具体的编写规范。

本文件对 GB/T7713-1987 中的学术论文编写内容进行了必要的检查、更新,进而形成单独的学术论文编写规则,代替GB/T7713一1987中的学术论文编写格式部分。
\end{preamble}
